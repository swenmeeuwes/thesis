\chapter{Betrokkenen}
\section{Stelian Paraschiv (Afstudeerbegeleider)}
Zal de student vanuit school begeleiden tijdens het afstuderen. Hij zal de planning bewaken en zo de voortgang van de student waarborgen. Tijdens de afstudeerperiode zal de afstudeerbegeleider twee keer langskomen bij het bedrijf om de voortgang van de student te bespreken en feedback te geven waar nodig.

\section{Ivo Swartjes (Bedrijfsbegeleider)}
Zal de student vanuit het bedrijf begeleiden tijdens het afstuderen. Hij zal zijn kennis betreft de bewerkers ter beschikking stellen. Verder zal hij de voortgang van de student bewaken en feedback geven op de producten van de student. Elke week zal er minimaal één gesprek tussen de bedrijfsbegeleider en de student plaatsvinden.

\section{Swen Meeuwes (Afstudeerstudent)}
Zal het afstudeeronderzoek uitvoeren zoals beschreven in het plan van aanpak. Hij zal onderzoek doen naar de verschillende (hoofd)onderwerpen binnen de afstudeeropdracht, waaruit een advies en eventuele prototypes zullen voortstromen. Dit advies en eventuele onderbouwende prototypes zal hij aanbieden aan de bedrijfsbegeleider ter evaluatie. Verder is hij verantwoordelijk voor de kwaliteit en stiptheid van zijn opleveringen en het initiatief om gesprekken met betrokkenen in te plannen. Tijdens het afstudeeronderzoek zal de student 4 dagen in de weken aanwezig zijn bij het bedrijf.

\section{Corporate Learning}
De ‘Corporate Learning’ afdeling van het bedrijf houdt zich vooral bezig met ‘narrative games’. De student zal (informele) interviews inplannen om met medewerkers binnen deze afdeling om kennis te vergaren om het gebied van de bewerkers. Hiernaast zal de student vanuit deze interviews requirements voor de nieuwe bewerkers kunnen opstellen.