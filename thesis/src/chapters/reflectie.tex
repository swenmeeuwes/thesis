\chapter{Persoonlijke reflectie}
\label{ch:reflection}
% \subsubsection{Afstuderen}
Het uitvoeren en documenteren van dit onderzoek heb ik ondervonden als zeer leerzaam. Dit onderzoek droeg bij aan mijn persoonlijke kennisontwikkeling op het gebied van verschillende technologieën, design patterns en content tooling bij narrative games. Voor dit onderzoek waren er drie doelen opgesteld (\autoref{sec:doelstelling}). Deze doelen kwamen voort uit de behoeftes van verschillende stakeholders met ieder een eigen perspectief. Zo had ik tijdens het onderzoek te maken met mijn begeleider vanuit school, een opdrachtgever vanuit het stagebedrijf en mijn eigen doelstelling; om meer te weten te komen over het onderwerp. Vanuit school werd er van mij verwacht dat ik mijn stagebegeleider up-to-date hield, maar tegelijkertijd verwachtte de opdrachtgever ook resultaat. Dit forceerde mij om het juiste balans te vinden tussen onderzoeken, documenteren en het werken aan een prototype.

\subsubsection{Complexiteit}
Ik was wel bekend met narrative games, maar had nog nooit zo diep in de gebruikte toolings gedoken. Een groot deel van het onderzoek was nieuw voor mij, vooral de termen 'formalisme' en 'overkoepelende projectstructuur'. Daarbij zijn narrative games een vrij niché game genre, wat het soms moeilijk maakte om informatie over het onderwerp te vergaren. 

Dit onderzoek houdt rekening met vele actoren. Voor het bedrijf is er rekening gehouden met zowel haar huidige situatie (huidige tech stack, narrative game template, software library, huidige editors) als gewenste toekomst situatie (collaborative editing, webomgeving). Wat betreft het schrijven van een narratief is er gekeken naar het onderscheid maken tussen content en formalismen die de richtlijnen zetten voor het definiëren van game content. Daarnaast is er ook aandacht besteed aan het inzetbaar maken van de tooling voor diverse projecten. Tenslotte komt alle game content samen in een overkoepelende projectstructuur om de foutgevoeligheid van de tooling te verlagen en nieuwe functionaliteiten toe te laten.

De combinatie van nieuw terrein en het rekening houden met vele actoren zorgde voor een vrij complex afstudeeronderzoek.

\subsubsection{Achteraf}
Tijdens het schrijven van mijn mandaat was ik erg druk bezig met mijn minor. Hierdoor kon ik slechts één keer kort afspreken met de opdrachtgever en bedrijfsbegeleider. In dit gesprek zijn we tot een onderwerp gekomen, waaruit eventuele interessante gebieden volgden. Deze zijn omgezet tot deelvragen in mijn mandaat. Aan het begin van mijn stageperiode had ik opnieuw gesprekken ingepland met mijn opdrachtgever en bedrijfsbegeleider. Tijdens deze gesprekken werd er gezocht naar de vraag achter de vraag, waaruit de uiteindelijke deelvragen zijn gekristalliseerd. Het was noodzakelijk om mijn deelvragen aan te passen wat resulteerde in tijdsverlies. Hoewel ik tijd voor deze oriëntatie had ingepland \autoref{app:ganttchart} liep het toch een beetje uit. Hieronder heeft, voor mijn gevoel, de documentatie van sommige hoofdstukken geleden. Wel was het aanpassen van de deelvragen nodig voor het onderzoek, deze sluiten nu beter aan bij de centrale onderzoeksvraag. Ik ben blij dat ik deze stap genomen heb. Door continu de benodigde tijd te inventariseren (zoals in \autoref{app:timecostplanning}) wist ik waaraan ik meer of minder tijd kon besteden om op planning te blijven.

Bij een volgend onderzoek zou ik vooral de structuur van mijn verslag anders aanpakken. Door een fasering toe te passen kan het probleem en proces (van probleem tot oplossing) meer naar voren komen. Ik merkte dat het lastig was om op een chronologische wijze mijn onderzoek te documenteren. Waarschijnlijk maakt dit het afstudeerverslag ook lastiger te lezen.

Het tijdig opzetten en gebruik maken van \LaTeX was een goede keuze. Ik kon mij focussen op het schrijven en mij later druk maken om de opmaak en citaat formaat. Verder was Mendeley\footnote{https://www.mendeley.com/} een gigantische hulp in het beheren van mijn bronnen. Hierin kon ik bronnen ordenen en automatisch een bibliografie voor \LaTeX en dus mijn afstudeerverslag laten genereren. Wel had ik bij naderinzien wel iets meer tijd kunnen inplannen voor het documenteren van mijn onderzoek.

\subsubsection{Afsluiting}
Dit onderzoek heeft mij motivatie gegeven om mijn zelfstandige narrative game project weer op te pakken. Het kan mij potentieel helpen om een deel van het groter probleem op te lossen; het definiëren van content in digitale interactieve verhalen. Met de verkregen inzichten kan ik toewerken naar een soort gelijke tool en weer nieuw leven blazen in mijn project.