\chapter{Doelstelling}
Het bedrijf hoopt na zes maanden te beginnen met het ontwikkelen van een nieuwe narratieve omgevingsbewerker, zodat ze efficiënter en goedkoper producten kunnen opleveren aan de klant. Hiervoor is het belangrijk om binnen de zes maanden zoveel mogelijk kennis en ervaring te verzamelen. 
In de oriëntatiefase van het onderzoek zal onder andere de wensen en eisen van de nieuwe bewerkers in kaart worden gebracht. Deze requirements zullen worden besproken met de bedrijfsbegeleider en zal samen met de student de scope verkleinen. De scope zal requirements bevatten waarnaar de student onderzoek zal doen. Dit onderzoek kan de student eventueel onderbouwen met prototypes.
Door vooraf onderzoek te doen naar deze requirements zal het ontwikkelproces van de nieuwe bewerkers in de toekomst efficiënter verlopen. Het onderzoek dient ter ondersteuning van de uiteindelijke implementatie van de bewerkers.
Het bedrijf zal van de student een scriptie met daarin het advies over de besproken requirements ontvangen. Hiernaast zal het bedrijf een prototype en de bijbehorende broncode van de adviezen ontvangen.
