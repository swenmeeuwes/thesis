\chapter{Probleemstelling}
Het bedrijf heeft de bewerkers in 2008 opgezet \cite{interviewivo}. Echter zijn over de jaren heen de eisen en wensen van deze bewerkers veranderd. Omdat de bewerkers zijn gemaakt in een verouderd framework is het moeilijk om aanpassingen door te voeren. Het niet vernieuwen van deze bewerkers zal ten koste van efficiëntie gaan. Narratieve projecten zullen meer tijd kosten om te ontwikkelen en zullen dus duurder worden voor de klant. Hiernaast roept het frustratie op bij de gebruikers van de bewerkers, omdat de bewerkers in hakken op hun efficiëntie.

\section{(Hoofd)activiteiten}
\begin{itemize}
    \item De huidige situatie zal in kaart worden gebracht.
    \begin{itemize}
        \item Waarom zijn de bewerkers nodig?
        \item Welke rol gaan de bewerkers spelen in het ontwikkelproces?
        \item Wie gaan de bewerkers gebruiken?
        \item Hoe worden de bewerkers gebruikt? Wat zijn hierbij de wensen van de gebruikers, waarom is dit wenselijk?
    \end{itemize}

    \item Er zal ingezoomd worden op specifieke vraagstukken ter voorbereiding van de bewerkers die het bedrijf later zal gaan ontwikkelen.
    \item Er zullen informele interviews worden afgenomen met medewerkers binnen de ‘Corporate Learning’ afdeling om vraagstukken te identificeren en mogelijk advies te evalueren.
    \item Er zal onderzoek gedaan worden naar de specifieke vraagstukken (in overleg met de bedrijfsbegeleider), waaruit een advies en eventueel prototype zal voortstromen.
\end{itemize}

\section{Afbakening}
\begin{itemize}
    \item Er zal niet specifiek gekeken worden naar de usability rondom de bewerker(s).
    \item Er zal nauw samengewerkt worden met de afdeling ‘corporate learning’.
    \item Er zal geen vervangend product worden opgeleverd. De opgeleverde prototypes zullen zich focussen op specifieke vraagstukken van de bewerkers en dienen slechts ter onderbouwing van het onderzoek advies.
\end{itemize}