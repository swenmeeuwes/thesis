\chapter{Inleiding}

% Aanleiding
Rond 2008 begon \organisation{} narratieven te verwerken in hun serious games om op
verhalende wijze gedragsverandering toe te passen \cite{interviewivo}. Om game designers deze
narratieven te laten definiëren zijn er twee bewerkers opgezet; één voor de verhaallijn en één voor de dialogen die plaats vinden in deze verhaallijn. De huidige
versies van deze bewerkers zijn gemaakt met behulp van de Apache Flex SDK
en ActionScript3 \cite{interviewivo}. Over de jaren heen zijn de verwachtingen van de bewerkers veranderd, maar ze zijn niet tot weinig uitgebreid omdat de achterliggende
softwarearchitectuur niet schaal- en houdbaar is \cite{interviewivo}. % Waarom is deze niet schaalbaar?

% Belang

% Het bedrijf, werk omgeving en taken

% De doelstelling
Het bedrijf hoopt na zes maanden te beginnen met het ontwikkelen van een nieuwe bewerker om narratieven in hun serious games te structuren en definiëren.
Deze nieuwe beweker(s) moeten zorgen voor een efficiëntere workflow waardoor het bedrijf voor lagere kosten hun producten kunnen opleveren aan de klant.
Hiervoor is het belangrijk om binnen de zes maanden zoveel mogelijk kennis en ervaring te verzamelen. % op het gebied van? Tooling? Visual scripting? Achterliggende structuur?
Verder kan er nagedacht worden over mogelijke oplossingen op problemen die voort komen uit het onderzoek zodat deze het ontwikkelproces later niet zullen hinderen.

% De probleemstelling -> centerale onderzoeksvraag en deelvragen
Voor het definiëren van dialogen in \emph{narrative games} gebruikt \organisation{} verouderde bewerkers die gemaakt zijn met behulp van de \href{https://en.wikipedia.org/wiki/Apache_Flex}{Apache Flex SDK} in \href{http://www.adobe.com/devnet/actionscript/articles/actionscript3_overview.html}{ActionScript3} met \href{https://en.wikipedia.org/wiki/Adobe_Flash_Builder}{Adobe Flash Builder} als \emph{integrated development environment}.
Echter werken er nog weinig programmeurs bij \organisation die kennis hebben van \href{https://en.wikipedia.org/wiki/Apache_Flex}{Apache Flex} en \href{http://www.adobe.com/devnet/actionscript/articles/actionscript3_overview.html}{ActionScript3}.
Verder wekt de architectuur en beperkte schaalbaarheid van de bewerkers frustratie op bij de game developers en game designers.
Dit zorgt ervoor dat het steeds lastiger wordt om deze bewerkers te onderhouden en uit te breiden.
Projecten verschillen in features en content, maar de huidige bewerkers maken het moeilijk om deze aspecten te splitsen per project.
Hierdoor zitten er veel features in de bewerkers die maar één keer nodig waren en nu de bewerkers bevuilen.
Gebruikers hebben door deze bevuiling steeds minder overzicht.
Dit alles zorgt voor een daling in efficiëntie en bekommerd innovatie.
De gewenste situatie is om te beschikken over een overzichtelijke narratieve omgevingsbewerker met een schaalbare en houdbare architectuur.
In deze vernieuwde narratieve omgevingsbewerker kunnen er makkelijk nieuwe features en content worden toegevoegd.
Verder kan de bewerker worden ingericht per project om vervuiling te voorkomen.
Vervolgens kunnen game developers content integreren zonder deze te hardcoden in de bewerker(s).
