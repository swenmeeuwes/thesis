\chapter{Conclusie}
\label{ch:conclusion}
Dit onderzoek koppelt terug op de centrale onderzoeksvraag: 
\begin{quote} 
    \centering
    \large
    \textit{
        "Hoe kan er een flexibele tool worden opgezet voor het vertellen van diverse digitale interactieve verhalen?"
    }
\end{quote}

Uit \autoref{ch:technologystack} blijkt dat het aanpassen van de huidige technology stack cruciaal is voor de toekomst van de editor. De huidige tech stack bevat componenten met een onzekere toekomst, kleine community en door het gebruik van meerdere programmeertalen zijn er overtollige libraries ontstaan. Bij het ontwikkelen van een nieuwe editor is het cruciaal dat de ondersteunende tech stack toekomstgericht is, om de levensduur van de nieuwe editor te verhogen. Hoewel \&ranj voorlopig gebruik wilt blijven maken van desktopapplicaties willen ze in de toekomst toe werken naar een web omgeving die collaborative editing, het met meerdere gebruikers tegelijk aan het narratief kunnen werken, ondersteund. Toekomstgerichtheid en een grote community zijn beide goede redenen om Electron te gebruiken als ontwikkelplatform voor de nieuwe editors. Electron is een ontwikkelplatform waarin desktopapplicaties ontwikkelt kunnen worden door middel van webtechnologieën. Het ontwikkelplatform heeft zichzelf bewezen in applicaties zoals ‘Slack’, ‘Discord’, ‘Atom’ en ‘Visual Studio Code’. Daarbij vergroot dit de herbruikbaarheid van de editors wanneer het bedrijf toe gaat werken naar een webapplicatie. Naast de gigantische community rondom webtechnologieën integreert Electron ook goed in de huidige tech stack die al gebruik maakt van webtechnologieën. Om de user interface synchroon te houden met de staat van de applicatie kan gebruik worden gemaakt van React. Hiermee kan de user interface onderverdeelt worden in kleinere componenten. Tenslotte brengt de community rondom web technologie bestaande oplossingen voor het visualiseren van diagrammen met zich mee. JointJS + Rappid en MxGraph beschikken beide aan een rijke feature set waarmee diagrammen zoals in de huidige editors gevisualiseerd kunnen worden. JointJS + Rappid heeft een betere documentatie en blijkt de populairdere optie te zijn van de twee. 

In \autoref{ch:diversiteitingamecontent} wordt er een voorstel gedaan op het flexibel omgaan met diversiteit in game content. Door gebruik te maken van JSON-schema kunnen content typen toegevoegd, aangepast en verwijderd worden zonder de editor opnieuw te hoeven compileren. Hiernaast kan er voor ieder project een JSON-schema worden opgesteld met project specifieke content typen, waardoor de selectie aan content typen focus behoud. Het gebruiken van de schema’s in de editor vereist enkele voorbereidingsstappen die eenmalig, bij het opstarten van de editor, uitgevoerd moeten worden.

De scheiding tussen de editor en formalismen wordt gelegd in \autoref{ch:formalism}. Door de representatie van syntactische vormen los te koppelen met behulp van het builder pattern kunnen deze syntactische vormen gebruikt worden in meerdere formalismen. Door middel van validatie functies op zowel groeps- als individueel niveau kunnen er restricties worden opgelegd aan het verbinden en het in elkaar voegen van syntactische vormen. Met het visitor pattern kan de visuele boom in de editor om gezet worden in een export-bestand volgens de richtlijnen van het vastgestelde formalisme. Het loskoppelen van formalisme bevorderd de flexibiliteit van de editor en maakt het samenvoegen van de story- en dialog editor mogelijk.

Zoals beschreven in \autoref{ch:overkoepelendeprojectstructuur} worden er assets in het verhaal verwerkt. Door een overkoepelende projectstructuur te implementeren kan game content met elkaar worden verbonden en geordend. Hiervoor moet de editor beschikbare assets beheren en semantiek binden aan de assets door middel van de bestandsextensie van de asset. Voor een entiteiten laag is er een user interface nodig waarin de gebruiker assets kan koppelen aan een entiteit wat verder onderzoek vereist. Tenslotte maakt deze overkoepelende projectstructuur het mogelijk om alles assets samen te bundelen in een archief. Dit maakt het laden van narrative games sneller.

Het gebruik van JSON-schema’s voor content typen, een scheiding leggen tussen de editor en formalisme en het implementeren van een overkoepelende projectstructuur dienen allen als stap richting een flexibelere tool voor het vertellen van diverse digitale interactieve verhalen.
