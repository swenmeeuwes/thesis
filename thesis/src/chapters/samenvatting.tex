\chapter*{Samenvatting}
In ontwikkelprocessen wordt er veel gebruik gemaakt van tooling om efficiëntie te verhogen. Dit onderzoek focust zich op tooling bij het ontwikkelproces van narrative games. Deze tooling laat mensen zonder programmeerkennis content specificeren die later in de game getoond zal worden. Tooling wordt over het algemeen geacht flexibel te zijn, zodat deze inzetbaar is in meerdere situaties.
\begin{quote} 
    \centering
    \large
    \textit{
        "Hoe kan er een flexibele tool worden opgezet voor het vertellen van diverse digitale interactieve verhalen?"
    }
\end{quote}

\noindent In dit onderzoek wordt er een prototype opgezet die stappen maakt in de flexibiliteit van narrative tooling. Deze stappen worden gemaakt op het gebied van typen content, formalisme en een overkoepelende projectstructuur.

Het prototype is gebouwd met behulp van Electron, een ontwikkelplatform dat het gebruik van webtechnologieën voor het ontwikkelen van desktopapplicaties mogelijk maakt. Om de user interface synchroon te houden met de staat van de applicaties wordt er gebruik gemaakt van de React library. Tenslotte gebruikt het prototype de JointJS library om een visual scripting interface aan te bieden, zodat mensen zonder programmeerkennis de tool kunnen gebruiken.
Door middel van ‘content typen’ maakt het prototype onderscheid tussen verschillende soorten content. Dit zijn kleine datastructuren met een betekenis in de game. Door JSON-schema te implementeren kunnen deze content typen gespecificeerd worden zonder de tool opnieuw te hoeven compileren. Dit maakt deze tool flexibel en inzetbaar voor narrative games op een breedspectrum aan soorten content.

Het prototype legt een scheiding tussen formalisme en visuele componenten door gebruik te maken van het visitor design pattern. Het visitor pattern biedt context voor het trans compileren van visuele structuur naar dataformaat. Componenten worden gescheiden van hun representatie door het builder design pattern. Daarmee ondersteund de tool meerdere formalismen, zoals story graphs en interactive behaviour trees. Welke ieder gebruikt kunnen worden in een passende situatie. Dit maakt het onderhouden van meerdere tools met ieder zijn eigen formalisme overbodig en bevorderd de flexibiliteit van de tool.

De tooling voor narrative games heeft te maken met vele bestanden (afbeeldingen, video’s, geluiden) die gebruikt worden in het narratief. Vooral grotere narrative games zijn gesplitst in bestanden die naast het verhaal ook dialogen bevatten. Deze bestanden die samen de game content vormen worden ook wel assets genoemd. Verder maken narrative games gebruik van variabelen om conditionaliteit af te dwingen. In dit onderzoek worden er stappen gemaakt richting een overkoepelende projectstructuur waarin game content geordend en met elkaar verbonden wordt. Zo kunnen overkoepelende variabelen gebruikt worden in zowel het verhaal als dialoog. Indien mogelijk kunnen assets worden gepreviewed in de editor. Hiernaast kunnen bestanden worden verplaatst binnen de assetsfolder zonder dat referenties gebroken worden. Tenslotte kunnen project specifieke configuraties worden opgenomen in de overkoepelende projectstructuur. Dit maakt het inrichten van content typen mogelijk per project.

Het gebruik van JSON-schema’s voor content typen, een scheiding leggen tussen de editor en formalisme en het implementeren van een overkoepelende projectstructuur dienen allen als stap richting een flexibelere tool voor het vertellen van diverse digitale interactieve verhalen.

Dit onderzoek is uitgevoerd bij \&ranj, een serious game bedrijf dat zich specialiseert in narrative games. Met het maken van keuzes over ontwikkelplatformen wordt de huidige technology stack van \&ranj in kaart gebracht om te kijken of de keuze goed aansluit. Verder kan naast andere story telling tools ook \&ranj gebruik maken van deze implementaties om hun tooling flexibeler te maken.