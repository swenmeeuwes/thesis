\chapter{Aanleiding}
Sinds 1999 houdt het \emph{serious game} bedrijf \organisation{} zich al bezig met het toepassen van gedragsverandering op een speelse manier. Zij passen \emph{gamification} en \emph{serious games} toe om gedragsverandering te realiseren. Zo heeft het bedrijf een spel ontwikkeld dat jongeren bewust maakt van discriminatie en leert deze jongeren hiermee om te gaan \cite{fairplay}. Veel \emph{serious games} die het bedrijf ontwikkeld maken gebruik van narratieven; in het spel komt een sterk verhaal naar voren. Deze spellen plaatsen de speler in een veilige leer omgeving waarin ze dialogen met verschillende gesprekspartners aan gaan. In deze dialogen maken de spelers keuzes die resulteren in directe feedback. De directe feedback is een groot voordeel omdat spelers zich gauw bewust worden van hun keuzes. Hierdoor zien ze hun fouten gauw in en kunnen ze gelijk reflecteren. Dit resulteert in gedragsverandering.Om het product betaalbaar te maken en het ontwikkelproces efficient te laten verlopen heeft \organisation{} een format opgezet voor de narratieve spellen en zo enkele aspecten van dit product gestandaardiseerd. Er is een framework opgezet waarin de narratieven zich afspelen en er zijn bewerkers opgezet om deze narratieven te definiëren. Echter zijn de bewerkers opgezet in 2008 en zijn de wensen en eisen van deze bewerkers veranderd over de jaren heen \cite{interviewivo}. Het bedrijf wilt graag deze bewerkers vernieuwen om te voldoen aan de nieuwe eisen en de visie van het bedrijf.