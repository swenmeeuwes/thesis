\chapter{Persoonlijke reflectie}
\label{ch:reflection}
Het uitvoeren en documenteren van dit onderzoek heb ik ondervonden als zeer leerzaam. Niet alleen droeg dit onderzoek bij aan mijn persoonlijke kennisontwikkeling op het gebied van verschillende technologieën en design patterns, maar had ook drie doelen (zoals beschreven in \autoref{sec:doelstelling}). Deze doelen kwamen voort uit verschillende stakeholders met ieder een eigen perspectief. Zo had tijdens het onderzoek ik te maken met mijn begeleider vanuit school, een opdrachtgever vanuit het stagebedrijf en mijn eigen doelstelling; om meer te weten te komen over het onderwerp. Vanuit school werd er van mij verwacht dat ik mijn stagebegeleider up-to-date hield, maar tegelijkertijd verwachten de opdrachtgever ook resultaat. Dit forceerde mij om het juiste balans te vinden tussen onderzoeken, documenteren en het werken aan een prototype.

Verder blijkt het maar weer hoe belangrijk een uitgebreid vooronderzoek is. Tijdens het schrijven van mijn mandaat was ik erg druk bezig met mijn minor, waardoor ik weinig tijd had voor vooronderzoek. Hierdoor kon ik slechts één keer kort afspreken met de opdrachtgever en bedrijfsbegeleider waaruit deelvragen zijn opgesteld voor mijn mandaat. Aan het begin van mijn stageperiode had ik gesprekken ingepland met mijn opdrachtgever en bedrijfsbegeleider waaruit bleek dat er meer vragen rondom het onderwerp speelde. Het was noodzakelijk om mijn deelvragen aan te passen wat resulteerde in tijdsverlies. Deze tijdsdruk kon worden voorkomen door beter vooronderzoek te doen. Ik heb het gevoel dat de documentatie van sommige hoofdstukken erg onder deze tijdsdruk geleden heeft.

Het tijdig opzetten en gebruik maken van \LaTeX was een goede keuze. Ik kon mij focussen op het schrijven en mij later druk maken om de opmaak en citaatstyle. Verder was Mendeley\footnote{https://www.mendeley.com/} een gigantische hulp in het beheren van mijn bronnen. Hierin kon ik bronnen ordenen en automatisch een bibliografie voor \LaTeX en dus mijn afstudeerverslag laten genereren.

% tussen tijdse planningen