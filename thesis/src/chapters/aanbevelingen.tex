\chapter{Aanbevelingen}
\label{ch:aanbevelingen}
In dit hoofdstuk wordt er advies uitgebracht op het gebied van;

\subsubsection{De toekomst van de editor}
Het bedrijf wilt in de verre toekomst over een web omgeving beschikken waarin narratieven geschreven kunnen worden. Wanneer de nieuwe editor ontwikkeld wordt als een desktop applicatie is het van belang om een scheiding te leggen tussen desktop functionaliteiten (aangeboden door Node en Electron) en de rest van de applicatie. Dit zal de herbruikbaarheid van de editor bevorderen en is wenselijk wanneer het bedrijf de editors wilt verhuizen naar een web omgeving. Het ‘adapter design pattern’ zou kunnen helpen bij het leggen van de scheiding tussen een desktop- en een web omgeving.

\subsubsection{Diagramming libraries}
In \autoref{ch:technologystack} zijn er eisen opgesteld waaraan een diagramming library in de nieuwe editor moet voldoen. Dit resulteerde in meerdere geschikte libraries, waarvan twee het meeste potentie hebben. Het is aan te raden om zowel met JointJS+Rappid als MxGraph een snel prototype te realiseren, om te valideren dat deze passen in de nieuwe editor. Hiernaast zal er kritisch gekeken moeten worden naar de feature set van beide libraries, met name naar de ‘real-time collaboratie’ feature. De vraag hierbij is hoe belangrijk het is om over deze functionaliteit te beschikken.
