\chapter*{Samenvatting}
In ontwikkelprocessen wordt er veel gebruik gemaakt van tooling om efficiëntie te verhogen. Tooling wordt geacht om flexibel te zijn zodat deze inzetbaar is in meerdere situaties. Dit onderzoek focust zich op het ontwikkelproces achter narrative games. Om mensen zonder programmeerkennis content in de game te laten specificeren wordt het ontwikkelproces ondersteund door tooling. Omdat narrative games sterk verschillen in content wordt deze tooling geacht om flexibel te zijn, zodat deze inzetbaar is voor meerdere narrative game projecten.
\begin{quote} 
    \centering
    \large
    \textit{
        "Hoe kan er een flexibele tool worden opgezet voor het vertellen van diverse digitale interactieve verhalen?"
    }
\end{quote}

In veel van deze tools zit de interpretatie van de visuele componenten waaruit het verhaal wordt opgebouwd vastgebakken in de componenten zelf. Hierdoor dwingen deze tools één formalisme af en is het relatief moeilijk om andere formalismen te ondersteunen. Zo wordt men in de meeste story telling tools geforceerd om binnen de richtlijnen van een story graphs, waarbij het verhaal opgedeeld wordt in states, te werken. Er is een prototype opgezet waarin de scheiding tussen formalisme en visuele componenten wordt gelegd. Dit prototype biedt een interface voor het formaliseren van zogenaamde syntactische vormen, die bouwblokken worden genoemd, en het omzetten van een visuele structuur naar een dataformaat volgens de richtlijnen van het desbetreffende formalisme. Dit maakt het mogelijk om meerdere formalismen te ondersteunen en bevorderd de flexibiliteit van de desbetreffende editor.

Vooral grotere narratieven bestaan uit een verhaal met meerdere dialogen. Om conditionaliteit aan paden binnen het verhaal en dialogen te vinden wordt er gebruik gemaakt van variabelen. Deze variabelen moeten bruikbaar zijn op zowel verhaal als dialoog niveau. Verder worden er naast tekstuele inhoud ook visuals en media in het narratief verworven. Deze media moet in de tool getoond worden om inzicht te geven in het verloop van het verhaal. Door op plaatsen waar media gebruikt wordt een voorbeeldje te tonen van de gebruikte media kan de tool meer overzicht scheppen. Er wordt een voorstel gedaan op een basis voor een overkoepelende projectstructuur. Hierin brengt de tool overkoepelende constructen samen in en projectstructuur. Door bestanden in een folder te beheren en eventuele veranderingen in deze folder te verwerken beschikt de tools over een sterke referentie naar bestanden die gebruikt kunnen worden in het narratief. Er worden twee semantische lagen aan de bestanden gekoppeld, zodat er bepaalde typen bestanden afgedwongen kunnen worden in de tool, om zo fouten te voorkomen.

Tooling moet onderscheid maken tussen typen content. Een voorbeeld hiervan is ‘quiz content’ die bestaat uit een vraag en antwoorden, waarvan één antwoord juist is. Deze tekst moet door de game engine anders worden getoond dan ‘decision content’, hierbij wordt de speler ook geacht een keuze te maken, maar deze keuze is niet goed of fout. Narrative games zijn niet gebonden aan een bepaald onderwerp wat resulteert in een breedspectrum aan content. De ondersteunende tooling moet flexibel zijn om deze diversiteit aan content te ondersteunen, zodat deze inzetbaar is voor meerdere projecten. Dit betekend dat het makkelijk moet zijn om nieuwe typen content toe te voegen. Het prototype werkt met ‘content typen’, kleine datastructuurtje die een betekenis hebben in de game. Deze worden gespecificeerd in JSON-schema, zodat content typen makkelijk kunnen worden toegevoegd, aangepast of verwijderd. Voor ieder project kan er met de overkoepelende projectstructuur een JSON-schema worden opgezet die alleen content typen bevat die bruikbaar zijn voor het desbetreffende project.

Er wordt gekeken naar bijpassende technologieën als basis voor de tooling en hoe deze integreert met de huidige technologieën die \&ranj gebruikt. Ook wordt er rekening gehouden met het toekomstbeeld dat \&ranj bij deze tooling heeft. Als ontwikkelplatform is er gekozen voor Electron en NodeJS. Om user interface synchroon te houden met de staat van de applicatie wordt er gebruik gemaakt van React library. Deze keuzes hebben zich bewezen in veel gebruikte applicaties van Facebook, Microsoft, Discord en GitHub.

De tool wordt opgezet in Electron met de hulp van web technologieën. Door gebruik te maken van JSON-schema om content typen te specificeren kan de tool flexibel omgaan met diverse content. Het implementeren van een overkoepelende projectstructuur maakt het inrichten van content typen per project en het flexibel omgaan met assets mogelijk. Door de scheiding te leggen tussen formalisme en de tool kan de tool meerdere formalisme ondersteunen en inzetbaar zijn voor meerdere projecten. Dit dient allen als stap richting een flexibelere tool voor het vertellen van digitale interactieve verhalen.
