\chapter*{Samenvatting}
In ontwikkelingsprocessen wordt er veel gebruik gemaakt van tooling om efficiëntie te verhogen. Dit onderzoek focust zich op tooling rondom het specificeren van content in het ontwikkelingsproces van narrative games. Deze tooling laat mensen zonder programmeerkennis content specificeren die later in de game getoond zal worden. Door een visual scripting interface aan te bieden kunnen gebruikers nodes met elkaar te verbinden om zo content te definiëren voor het narratief.
Tooling wordt over het algemeen geacht flexibel te zijn, zodat deze inzetbaar is in meerdere situaties.
\begin{quote} 
    \centering
    \large
    \textit{
        "Hoe kan er een flexibele tool worden opgezet voor het vertellen van diverse digitale interactieve verhalen?"
    }
\end{quote}

\noindent In dit onderzoek worden er stappen gezet richting een flexibelere tool voor het specificeren van content achter digitale interactieve verhalen. Typen content worden dynamische gedefinieerd, zodat de tool afgestemd kan worden op het project. Door in de tooling bij nodes en de structuur de scheiding te leggen tussen interpretatie en representatie kan er worden gewerkt met formalismen (e.g. story graphs, interactive behaviour trees) die aansluiten bij het project. In een overkoepelende projectstructuur komen overkoepelende constructen, zoals gedeelde variabelen, en de dynamische game content samen. Er is een prototype opgezet om deze concepten te ondersteunen en valideren.
%
% Het prototype is gebouwd met behulp van Electron, een ontwikkelplatform dat het gebruik van webtechnologieën voor het ontwikkelen van desktopapplicaties mogelijk maakt. Om de user interface synchroon te houden met de staat van de applicaties wordt er gebruik gemaakt van de React library. Tenslotte gebruikt het prototype de JointJS library om een visual scripting interface aan te bieden, zodat mensen zonder programmeerkennis de tool kunnen gebruiken.

Door middel van ‘content typen’ maakt het prototype onderscheidt tussen verschillende soorten content. Dit zijn kleine datastructuren met een betekenis in de game. Door op een slimme manier gebruik te maken van JSON-schema kunnen deze content typen dynamisch worden gespecificeerd zonder de tool opnieuw te hoeven compileren. Dit maakt de tool flexibel en inzetbaar voor narrative games op een breed spectrum aan soorten content.

Het prototype legt een scheiding tussen formalisme en visuele componenten door gebruik te maken van het visitor design pattern. Het visitor pattern biedt context voor het transcompileren van visuele structuur naar dataformaat. Componenten worden gescheiden van hun representatie door het builder design pattern. Daarmee ondersteund de tool meerdere formalismen, zoals story graphs en interactive behaviour trees. Welke ieder gebruikt kunnen worden in een passende situatie. Dit maakt het onderhouden van meerdere tools met ieder zijn eigen formalisme overbodig en bevordert de flexibiliteit van de tool.

Wanneer er twee applicaties worden gebruikt voor het definiëren van het narratief (e.g. één voor het verhaal en één voor de dialoog) is het niet mogelijk om gedeelde variabelen in zowel verhaal als dialoog te gebruiken. Daarnaast weten deze applicaties niks af van bestanden (afbeeldingen, video's, geluiden) die in het narratief verworven worden. Naast dat het refereren naar deze bestanden foutgevoelig is, kunnen er geen previews van deze bestanden worden getoond in de tools. In een overkoepelende projectstructuur beheert de tooling haar bruikbare bestanden en kunnen overkoepelende constructen zoals gedeelde variabelen en content typen schema's ingericht worden per project.

% De tooling voor narrative games heeft te maken met vele bestanden (afbeeldingen, video’s, geluiden) die gebruikt worden in het narratief. Vooral grotere narrative games zijn gesplitst in bestanden die naast het verhaal ook dialogen bevatten. Deze bestanden die samen de game content vormen worden ook wel assets genoemd. Verder maken narrative games gebruik van variabelen om conditionaliteit af te dwingen. In dit onderzoek worden er stappen gemaakt richting een overkoepelende projectstructuur waarin game content geordend en met elkaar verbonden wordt. Zo kunnen overkoepelende variabelen gebruikt worden in zowel het verhaal als dialoog. Indien mogelijk kunnen assets worden gepreviewed in de editor. Hiernaast kunnen bestanden worden verplaatst binnen de assetsfolder zonder dat referenties gebroken worden. Tenslotte kunnen project specifieke configuraties worden opgenomen in de overkoepelende projectstructuur. Dit maakt het inrichten van content typen mogelijk per project.

Het slimme gebruik van JSON-schema’s voor het specificeren van content typen, een scheiding leggen tussen de editor en formalisme en het implementeren van een overkoepelende projectstructuur dienen allen als stappen richting een flexibelere tool voor het vertellen van diverse digitale interactieve verhalen.

Dit onderzoek is uitgevoerd bij \&ranj, een serious game bedrijf dat zich specialiseert in narrative games. Met het maken van keuzes over ontwikkelplatformen wordt de huidige technology stack van \&ranj in kaart gebracht om te kijken of de keuze goed aansluit. Verder kan naast andere storytelling tools ook \&ranj gebruik maken van deze implementaties om hun tooling flexibeler te maken.