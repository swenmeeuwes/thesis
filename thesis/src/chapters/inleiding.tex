\chapter{Inleiding}

% todo: misschien inzomen op behavioral change model 
\section{Aanleiding}
Het \emph{serious games} en \emph{gamification} bedrijf \organisation{} begon rond 2008 narratieven te verwerken in hun games om op
verhalende wijze gedragsverandering toe te passen \cite{interviewivo}. Tot op vandaag de dag passen ze nog steeds narratieven toe in hun games als middel om gedragsverandering te realiseren.
Om producten efficiënt te realiseren heeft \organisation{} dit narratieve formaat gestandaardiseerd. Deze standaardisatie maakt mogelijk om tooling op te zetten voor projecten die zich bevinden in deze scope. Zo zijn er voor het definiëren van narratieven twee bewerkers opgezet; één voor de verhaallijn en één voor de dialogen die plaats vinden in deze verhaallijn. Game designers binnen het bedrijf gebruiken deze tools om hun concept te realiseren.
% todo: Deze gedragsverandering wordt uitgelegd in het behavioural change model. {Daarop inzoomen en uitleggen in welk deel de narratieven zitten, echter is dit misschien net iets te ingewikkeld}

De huidige versies van deze bewerkers zijn gerealiseert met behulp van de Apache Flex SDK en ActionScript3. Over de jaren heen zijn de verwachtingen van de bewerkers veranderd, maar ze zijn niet tot weinig uitgebreid omdat de achterliggende softwarearchitectuur slecht schaal- en houdbaar word ondervonden \cite{interviewivo}.


\section{Belang}


\section{Opdrachtgever}


\section{Doelstelling}
Het bedrijf hoopt na zes maanden te beginnen met het ontwikkelen van een nieuwe bewerker om narratieven in hun serious games te structuren en definiëren.
Deze nieuwe beweker(s) moeten zorgen voor een efficiëntere workflow waardoor het bedrijf voor lagere kosten hun producten kunnen opleveren aan de klant.
Hiervoor is het belangrijk om binnen de zes maanden zoveel mogelijk kennis en ervaring te verzamelen. % op het gebied van? Tooling? Visual scripting? Achterliggende structuur?
Verder kan er nagedacht worden over mogelijke oplossingen op problemen die voort komen uit het onderzoek zodat deze het ontwikkelproces later niet zullen hinderen.

\section{Probleemstelling}
% De probleemstelling -> centerale onderzoeksvraag en deelvragen
Voor het definiëren van dialogen in \emph{narrative games} gebruikt \organisation{} verouderde bewerkers die gemaakt zijn met behulp van de \href{https://en.wikipedia.org/wiki/Apache_Flex}{Apache Flex SDK} in \href{http://www.adobe.com/devnet/actionscript/articles/actionscript3_overview.html}{ActionScript3} met \href{https://en.wikipedia.org/wiki/Adobe_Flash_Builder}{Adobe Flash Builder} als \emph{integrated development environment}.
Echter werken er nog weinig programmeurs bij \organisation die kennis hebben van \href{https://en.wikipedia.org/wiki/Apache_Flex}{Apache Flex} en \href{http://www.adobe.com/devnet/actionscript/articles/actionscript3_overview.html}{ActionScript3}.
Verder wekt de architectuur en beperkte schaalbaarheid van de bewerkers frustratie op bij de game developers en game designers.
Dit zorgt ervoor dat het steeds lastiger wordt om deze bewerkers te onderhouden en uit te breiden.
Projecten verschillen in features en content, maar de huidige bewerkers maken het moeilijk om deze aspecten te splitsen per project.
Hierdoor zitten er veel features in de bewerkers die maar één keer nodig waren en nu de bewerkers bevuilen.
Gebruikers hebben door deze bevuiling steeds minder overzicht.
Dit alles zorgt voor een daling in efficiëntie en bekommerd innovatie.
De gewenste situatie is om te beschikken over een overzichtelijke narratieve omgevingsbewerker met een schaalbare en houdbare architectuur.
In deze vernieuwde narratieve omgevingsbewerker kunnen er makkelijk nieuwe features en content worden toegevoegd.
Verder kan de bewerker worden ingericht per project om vervuiling te voorkomen.
Vervolgens kunnen game developers content integreren zonder deze te hardcoden in de bewerker(s).
